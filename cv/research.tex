\begin{rubric}{Research Experience}
  %
  %\subrubric{Post-Doctoral}
  %\entry*[Oct 2024 - \textit{present}]
  %\textbf{University of Vermont} - Burlington, VT.
  %\textit{Post-Doctoral Trainee}
  %
  %Advisor: Nimrat Chatterjee
  %\vspace{-0.5em}
  %\begin{itemize}
  %  \itemsep-0.5em
  %\item Studying the mechanism of REV1 in translesion synthesis of cancer cells and
  %  developing potential drug targets
  %\end{itemize}
  %
  \subrubric{Graduate}
  \entry*[Sept 2016 - June 2022]
  \textbf{University of California, Irvine} - Irvine, CA.
  \textit{Graduate Researcher, Chemistry}
  
  Advisor: Filipp Furche
  \vspace{-0.5em}
  \begin{itemize}
    \itemsep-0.5em
  \item[] \hspace{-0.33in}\textbf{Theory and Development}
  \item Derived and developed the adiabatic connection symmetry adapted perturbation theory
    (AC-SAPT) to understand the behaviors of noncovalent interactions (NIs)
  \item Applied AC-SAPT framework to diagnose and determine computational methods that can accurately
    predict NIs
  \item Developed multivariate AC-SAPT framework establishing the dispersion
    size-consistency condition
  \item[] \hspace{-0.33in}\textbf{Application-based projects}
  \item Communicated with experimentalists from the Vanderwal Lab at UCI and modelled
    the $\sim$200,000 atoms ribosome-drug interaction via \textit{in silico} which has led to
    potential drug candidates; publication \textit{in prep}
  \item Provided computational models for Long Group at UC Berkeley to understand the
    electronic structure of dilanthanide single molecule magnets; published in \textit{J. Am. Chem. Soc.}
  \item Developer for the \texttt{TURBOMOLE} quantum package suite and collaborated with \texttt{TURBOMOLE}
    developers worldwide via Git version control
  \item Contributed code that analyzes the density errors of electronic structure methods
  \end{itemize}
  %
  \subrubric{Undergraduate}
  \entry*[Jun 2015 - Sept 2015]
  \textbf{University of California, Irvine} - Irvine, CA.
  \textit{Undergraduate Researcher, Mathematics}

  Advisor: Frederic Y. Wan
  \vspace{-0.5em}
  \begin{itemize}
  \itemsep-0.5em
  \item 1 of 20 students accepted into the Mathematical and Computational Biology for Undergraduate
    summer program
  \item Engaged with mathematician to develop a dynamic kinetic model that predicted the early
    development of fruit flies matching experimental studies
  \end{itemize}
  %
  \entry*[Mar 2014 - Jun 2016]
  \textbf{University of California, Irvine} - Irvine, CA.
  \textit{Undergraduate Researcher, Biology}

  Advisor: Thomas L. Poulos
  \vspace{-0.5em}
  \begin{itemize}
  \itemsep-0.5em
  \item Simulated and predicted the mechanism of \textit{Leishmania major} peroxidase through
    molecular dynamics (MD) simulations; results supported experiments and published in
    \textit{J. Chem. Info. Model.}
  \item Predicted the dominant protein conformation of cytochrome P450 through
    MD simulations matching experiments; published in \textit{Proc. Natl. Acad. Sci. U.S.A.}
  \end{itemize}
  %
  \entry*[Oct 2013 - Jun 2016]
  \textbf{University of California, Irvine}
  \textit{Undergraduate Researcher, Chemistry}

  Advisor: Filipp Furche
  \vspace{-0.5em}
  \begin{itemize}
  \itemsep-0.5em
  \item Supported Prescher Lab with computational models to produce luciferin derivatives that emit
    $\sim$2x stronger signal for bioluminescence; published in \textit{ChemBioChem}
  \item Developed up to $\sim$4x faster algorithm for molecular property in the excited
    state and contributed code to the \texttt{TURBOMOLE} quantum package; published in
    \textit{J. Chem. Phys.}
  \end{itemize}
  %
  %
\end{rubric}
